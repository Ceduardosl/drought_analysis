\section{Introduction}
	As secas são fenômenos recorrentes nas mais diversas regiões do planeta, sendo capazes de produzir um amplo impacto multisetorial dado o nexo existente entre recursos hídricos e os mais diversos setores econômicos e sociais que compõem a sociedade moderna \citep{AghaKouchak2021,Mishra2007,Mishra2010,Wang2023,VanLoon2016,Xu2019}. O fenômeno da seca pode ser classificado em diferentes tipos, segundo as diferentes variáveis hidrometeorológicas e atividades humana-biologicas, são elas: seca meteorológica, agrícola, hidrológica, socioeconomica e ecologica.

	Como apontado por \citet{Wang2023,Xu2019}, pode-se ordernar esses diferentes tipos de seca segundo os processos hidroclimatológicos existentes no ciclo hidrológico de uma bacia hidrográfica. Precipitações abaixo da média por um certo período de tempo (Seca meteorolgica) podem induzir uma diminuição da umidade disponível do solo atrapalhando a produção agrícola, dando origem a uma seca agrícola. Se a seca meteorológica persistir por tempo suficiente, pode-se observar uma diminiução das vazões e reservas superficiais de água, originando uma seca hidrológica. Dada essa diminuição da disponbilidade hídrica, pode-se comprometer o atendimento de demandas consuntivas de água (seca socioeconomica) e de demandas biológicas (seca ecologica). Define-se como propagação de seca a transição entre os seus diferentes estados, tal como descrito acima.

	O fenômeno da seca é frequentemente definido como um extremo climático natural, ideia reforçada pela propagação de seca partindo de um déficit de precipitação que muitas vezes é originado pela variabilidade climática natural. Todavia, devido ao crescente impacto das atividades antrópicas nos ciclos naturais do planeta, dentre eles o ciclo hidrológico, uma nova definição de seca vem sendo amplamente discutida: the anthropic droughts.